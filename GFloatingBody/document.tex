% 导言区
\documentclass{article}

\usepackage{ctex}
\usepackage{graphicx}

%标题、作者及日期
\title{浮动体}
\author{张三丰}
\date{\today}

\graphicspath{}%图片在当前目录下

% 正文区(文稿区)
\begin{document}
	\maketitle
	\LaTeX{}中的浮动体:
	% 创建以figure包围的浮动体,[htbp]可选参数设定浮动体的排版位置
	% h,此处  t,页面顶部  b,页面底部  p,独立一页
	\begin{figure}[h]
		%让内容居中
		\centering
		\includegraphics[scale=0.3]{Koala}
		%通过caption命令设定插图的标题
		\caption{这是一只考拉}
	\end{figure}
	
	当然,在\LaTeX{}也可以使用表\ref{tab-score}所示的表格。
	
	% 创建以table包围的浮动体
	\begin{table}[h]
		%让内容居中
		\centering
		%定义标签完成交叉引用,目的是实现软编码,增加修改的灵活性
		\caption{考试成绩单}\label{tab-score}
		\begin{tabular}{| l || c | c | c | r|}
			\hline %表示横线
			姓名 & 语文 & 数学 & 外语 & 备注 \\
			\hline \hline %表示双横线
			李四 & 75 & 64 & 52 & 补考另行通知 \\
			\hline
			王五 & 80 & 82 &78 & \\
			\hline
		\end{tabular}
	\end{table}

\end{document}